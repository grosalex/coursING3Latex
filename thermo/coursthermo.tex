\documentclass[a4paper,10pt]{report}
\usepackage[utf8]{inputenc}

%opening
\title{Cours de thermodynamique ING3}
\author{Bruneau Alexandre}

\begin{document}

\maketitle

\begin{abstract}

\end{abstract}
\chapter{Base de la thermodynamique}
\section{Présentation}
Je skip parce que c'est des exemples lohol.

\section{Etat de la matière}
\subsection{Corps pur}
\paragraph{Def : }
Un corps pur est un constituant unique défini par sa formule chimique.

\subsection{Etat de la matière}
\paragraph{Phase solide:}
\begin{enumerate}
 \item Etat compact : les molécules sont très liés entre elles
 \item Distance interparticulle faible => cohérence et rigidité
 \item Quasi incompréssible et peu dilatable
\end{enumerate}
\paragraph{Phase Liquide:}
\begin{enumerate}
 \item Les molécules sont relativement liées
 \item Volume propre mais peut s'ecouler
 \item Quasi incompréssible et peu dilatable
\end{enumerate}
\paragraph{Phase Gazeuse :}
\begin{enumerate}
\item Etat dispersé de la matière molécules quasi pas liées entre elles
\item Les molécules sont soumises à l'agitation thermique
\item Se dilate et se comprime
\end{enumerate}

\subsection{Etat condensé :}
\paragraph{Def :}
Etat condensé : phase solide ou liquide. Caractérisé par :

Masses volumiques :
\begin{math}
 \rho = \frac{m}{V}
\end{math}

\begin{math}
 \rho_{eau}= 10^3 kg.m^3
\end{math}

densite :
\begin{math}
 d = \frac{\rho}{\rho_{eau}}
\end{math}

\subsection{Etat fluide :}
\paragraph{Def : }Etat fluide : gaz (compressible) ou liquide (incompréssible)

or 
\begin{math}
 \rho_{air}=1,3 kg.m^{-3}
\end{math}
 et pour un gaz on a 
 \begin{math}
  d=\frac{\rho}{\rho_{air}}
 \end{math}

 \paragraph{Quantité molaire :}
 \begin{math}
  n=\frac{N}{Na}
 \end{math}

 \paragraph{Relation des gazs parfaits :}
 \begin{math}
  PV=nRT
 \end{math}
\section{Echelle d'étude}
\begin{enumerate}
 \item Echelle microscopique => atome
 \item Echelle macroscopique => au niveau du système
 \item Echelle mesoscopique => echelle intermediaire
\end{enumerate}
\section{Grandeur Thermodynamique}
\subsection{Vocabulaire}
Un système thermodynamique est défini par une surface réelle ou fictive.
\begin{enumerate}
 \item Système fermé : Pas d'échange de matière avec l'extérieur.
 \item Système ouverte : echange de matière possible.
 \item Système adiabatique : Pas d'échange thermique
 \item Système isolé : Pas d'échange d'énergie ou de matière avec l'extérieur.
\end{enumerate}

\end{document} 
