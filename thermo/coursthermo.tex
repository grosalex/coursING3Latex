\documentclass[a4paper,10pt]{report}
\usepackage[utf8]{inputenc}

%opening
\title{Cours de thermodynamique ING3}
\author{Bruneau Alexandre}

\begin{document}

\maketitle

\begin{abstract}

\end{abstract}
\chapter{Base de la thermodynamique}
\section{Présentation}
Je skip parce que c'est des exemples lohol.

\section{Etat de la matière}
\subsection{Corps pur}
\paragraph{Def : }
Un corps pur est un constituant unique défini par sa formule chimique.

\subsection{Etat de la matière}
\paragraph{Phase solide:}
\begin{enumerate}
 \item Etat compact : les molécules sont très liés entre elles
 \item Distance interparticulle faible => cohérence et rigidité
 \item Quasi incompréssible et peu dilatable
\end{enumerate}
\paragraph{Phase Liquide:}
\begin{enumerate}
 \item Les molécules sont relativement liées
 \item Volume propre mais peut s'ecouler
 \item Quasi incompréssible et peu dilatable
\end{enumerate}
\paragraph{Phase Gazeuse :}
\begin{enumerate}
\item Etat dispersé de la matière molécules quasi pas liées entre elles
\item Les molécules sont soumises à l'agitation thermique
\item Se dilate et se comprime
\end{enumerate}

\subsection{Etat condensé :}
\paragraph{Def :}
Etat condensé : phase solide ou liquide. Caractérisé par :

Masses volumiques :
\begin{math}
 \rho = \frac{m}{V}
\end{math}

\begin{math}
 \rho_{eau}= 10^3 kg.m^3
\end{math}

densite :
\begin{math}
 d = \frac{\rho}{\rho_{eau}}
\end{math}

\subsection{Etat fluide :}
\paragraph{Def : }Etat fluide : gaz (compressible) ou liquide (incompréssible)

or 
\begin{math}
 \rho_{air}=1,3 kg.m^{-3}
\end{math}
 et pour un gaz on a 
 \begin{math}
  d=\frac{\rho}{\rho_{air}}
 \end{math}

 \paragraph{Quantité molaire :}
 \begin{math}
  n=\frac{N}{Na}
 \end{math}

 \paragraph{Relation des gazs parfaits :}
 \begin{math}
  PV=nRT
 \end{math}
\section{Echelle d'étude}
\begin{enumerate}
 \item Echelle microscopique => atome
 \item Echelle macroscopique => au niveau du système
 \item Echelle mesoscopique => echelle intermediaire
\end{enumerate}
\section{Grandeur Thermodynamique}
\subsection{Vocabulaire}
Un système thermodynamique est défini par une surface réelle ou fictive.
\begin{enumerate}
 \item Système fermé : Pas d'échange de matière avec l'extérieur.
 \item Système ouverte : echange de matière possible.
 \item Système adiabatique : Pas d'échange thermique
 \item Système isolé : Pas d'échange d'énergie ou de matière avec l'extérieur.
\end{enumerate}
\subsection{Equilibre thermodynamique}
\paragraph{Def :}
Dans un système à l'equilibre thermodynamique les grandeur définis à l'echelle mesoscopique sont constantes.
\paragraph{Propriété:}
Tout système isolé tends vers un état d'équilibre.

\subsection{Equation d'état}
\subsubsection{Paramètre d'état}
\paragraph{Def:}
Des paramètres d'états sont des grandeurs macroscopique définissant le système et susceptible d'etre modifié lors d'une transformation.

\subsubsection{Equation d'état}
\paragraph{Def :}
La relation reliant les différents paramètres d'état est l'équation d'état.

\subsubsection{Notion de phase}
\paragraph{Def :}
Une phase est une parti d'un système et les grandeurs sont des fonctions continues de l'espace.

\section{Le gaz parfait}
\subsection{Définition :}
C'est un modéle de comportement des gazs à basse pression.

On a : 
\begin{math}
 PV=nRT
\end{math}

On peut établir cette relation au niveau théorique en supposant :
\begin{enumerate}
 \item les particulles ponctuelles
 \item les particulles son sans intéractions entre elles
 \item les éléments sont élastique
\end{enumerate}
\subsection{Amélioration du modèle}
Modèle de Van der Waals :

\begin{equation}
 (P+\frac{n^2 a}{v^2})(V-nB)=nRT
\end{equation}
b est le volume occupé par les particulles et rends compte des interractions entre les particulles.

\section{Dilatation et compréssibilité d'un système}
Coefficient thermodynamique :

\begin{math}
 \alpha = \frac{1}{V} \frac{dV}{dt}
\end{math}

\begin{math}
 X_T = \frac{1}{V} \frac{\partial V}{\partial t}
\end{math}

\begin{math}
 \beta = \frac{1}{P} \frac{\partial P}{\partial T}
\end{math}

\chapter{Elements de statique des fluides}
\section{Introduction}
Il y a deux états pour les fluides :
\begin{enumerate}
 \item gaz
 \item liquide
\end{enumerate}
Hypothèse :
\begin{enumerate}
 \item On suppose les milieux contine
 \item On travaille au niveau mesoscopique
 \item Le fluide est supposé parfait
\end{enumerate}
\section{Relation de la statique des fluides}
\begin{equation}
  \overrightarrow{grad}(P) = \mu \overrightarrow{g}
\end{equation}
Pour la démo voir le cours.

\section{Cas des fluides incompréssible}
\subsection{Présentation}
ici on pose 
\begin{math}
 \mu(M)=\mu_0
\end{math}
On a alors 
\begin{math}
 \frac{d D}{d z} = \mu_0 g
\end{math}
\section{Fluide compréssible, Etudes des atmosphères}
\subsection{Présentation du modèles de l'atomosphère isotherme}


\begin{math}
 \mu = \mu(z) = constante
\end{math}

\begin{math}
 T=T_0=constante
\end{math}

\begin{math}
 g=constante
\end{math}

On a 
\begin{math}
 \frac{d P}{d z}=- \mu (z) g
\end{math}
or 
\begin{math}
 \mu(z) = \frac{m}{V} = \frac{nM}{V}
\end{math}
et 

\begin{math}
 PV= nrT_0 
\end{math}

\section{Poussé d'archimède}
\paragraph{Def:}Resultante des forces de pressions.
\begin{equation}
 \Pi=-\mu Vg
\end{equation}
\chapter{1 er principe de la thermodynamique}
\section{Présentation}
On étudie dans ce chapitre l`'evolution d'un système d'un état d'équilibre à un autre

\paragraph{Les transformations} peuvent etre :
\begin{enumerate}
 \item brusque : paramètres intensif ne sont alors pas définis pendant la transformation. Ils le sont lors des états initiaux et finaux.
 Les transformations sont irreverssibles.
 \item Lentes, on est alors tout au long de la transformation dans un état d'équilibre. On parle de transformation quasi statique.
 La transformation est réversible si on passe par les mêmes états d'équilibre dans un sens de la transformation ou dans l'autre. Cela implique
 l'absence de phénomène dissipatifs.
\end{enumerate}
\paragraph{Phénomènes dissipatifs}: ou causse d'irreverssibilités :
\begin{enumerate}
 \item frottement sec ou visqueux
 \item phénomène de diffusion
 \item Inhomogeinité des systèmes
 \item inelasticités
\end{enumerate}

\section{Energie d'un système}
\paragraph{Définition :}On a pour un système
\begin{equation}
 E = U + E_{c_{macro}} + E_{p_{ext}}
\end{equation}
\begin{equation}
 U = E_{c_{micro}} + E_{p_{interne}}
\end{equation}
\section{Transfert d'énéergie}
\subsection{Transfert thermique}
on note Q le transfert thermique au cours d'une transformation.

Il existe trois mode de transfert thermique :
\begin{enumerate}
 \item par conduction : louche en argent dans soupe
 \item par convection : mouvement de l'air dans l'amphi
 \item par rayonnement : chocolat au soleil
\end{enumerate}

\subsection{Travail des forces de pressions}
On considère un piston remplit d'un gaz. L'on applique une force extérieur sur le piston (on comprime le gaz).
Calculons \begin{math}
           W(\overrightarrow{F_{ext}})
          \end{math}
.

\begin{math}
 W(\overrightarrow{F_{ext}})= \int_x^{x+dx}\overrightarrow{F_{ext}} \overrightarrow{dl} = \int_x^{x+dx}\overrightarrow{F_{ext}} \overrightarrow{U_x} dx \overrightarrow{U_x} 
 =F_{ext} dx
\end{math}

or on a que 
\begin{math}
 W(F_{gaz\rightarrow piston})=- W(\overrightarrow{F_{ext}}) = -F_{ext} dx
\end{math}

et que : 
\begin{math}
 \overrightarrow{F_{piston\rightarrow gaz}}=-\overrightarrow{F_{gaz\rightarrow piston}} 
\end{math}

donc :
\begin{math}
 W(\overrightarrow{F_{piston\rightarrow gaz}}) = F_{ext} dx
\end{math}

On note \begin{math}
         \delta W = W_{x\rightarrow x+dx}
        \end{math} le travail élémentaire

Donc \begin{math}
      \delta W(\overrightarrow{F_{piston\rightarrow gaz}}) = F_{ext} dx = NRJ recu par le gaz
     \end{math}

\begin{math}
 \delta W_{recu par le gaz} = P_{ext} S dx = -P_{ext} dV
\end{math}
\paragraph{NB}: A l'equilibre mcanique on à :
\begin{math}
 P_{ext}* S = F_{ext} = P_{gaz}*S = P*S => P_{ext} = P
\end{math}

Si la transformation est quasi static alors on a \begin{math}
                                                  \delta W = -P dV
                                                 \end{math}

(je noterai la suite de cette partie plus tard)

\subsection{Autres forces de travail}
Avec une résistance electrique on aura :
\begin{math}
 \delta W = Ri^2dt=\frac{U^2}{R}dt
\end{math}
\subsection{1er principe}
\paragraph{Enoncé :}Pour un système fermé on a :
\begin{equation}
 \Delta U_{1\rightarrow2}= W_{1\rightarrow2} + Q_{1\rightarrow2}
\end{equation}
et \begin{math}
    d U = \delta W + \delta Q
   \end{math}
\paragraph{NB:}
Pour les gazs parfaits nous avons \begin{math}
                                   dU=C_VdT
                                  \end{math}
ou 

\begin{math}
C_V = 
\end{math}
capacite thermique a valeur constante

\paragraph{Exemple :}cf cours
\section{Fonction Entahalpie}
\paragraph{Definition :}est une fonction d'état extensive défini par 
\begin{math}
 H=U+PV
\end{math}

Et si on intégre on a 
\begin{math}
 \Delta H_{1\rightarrow2}= Q_{1\rightarrow2}
\end{math}
\section{Capacité thermique}
\paragraph{Loi de Joul :}Pour les gazs parfaits U et H sont uniquement fonction de T.
\paragraph{Définition}
\begin{math}
 C_V = \frac{\partial U}{\partial T}
\end{math}
et 
\begin{math}
 C_P = \frac{\partial H}{\partial T}
\end{math}

On pose \begin{math}
         \gamma=\frac{C_P}{C_V}
        \end{math}
et on a \begin{math}
         H=U+PV=U+nRT
        \end{math}

Donc l'on peut établir la relation de Mayer :
\begin{equation}
 C_P-C_V=nR
\end{equation}
\section{Loi de Laplace}
Pour un  gaz parfait, lors d'une transformation reversible ou seul le trvail des forces de pression interviennent on a :
\begin{equation}
 PV^\gamma=constante
\end{equation}
\begin{equation}
P^{1-\gamma}T^{\gamma} = constante 
\end{equation}
\begin{equation}
 TV^{\gamma-1}=constante
\end{equation}

\end{document} 
