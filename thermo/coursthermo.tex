\documentclass[a4paper,10pt]{report}
\usepackage[utf8]{inputenc}

%opening
\title{Cours de thermodynamique ING3}
\author{Bruneau Alexandre}

\begin{document}

\maketitle

\begin{abstract}

\end{abstract}
\chapter{Base de la thermodynamique}
\section{Présentation}
Je skip parce que c'est des exemples lohol.

\section{Etat de la matière}
\subsection{Corps pur}
\paragraph{Def : }
Un corps pur est un constituant unique défini par sa formule chimique.

\subsection{Etat de la matière}
\paragraph{Phase solide:}
\begin{enumerate}
 \item Etat compact : les molécules sont très liés entre elles
 \item Distance interparticulle faible => cohérence et rigidité
 \item Quasi incompréssible et peu dilatable
\end{enumerate}
\paragraph{Phase Liquide:}
\begin{enumerate}
 \item Les molécules sont relativement liées
 \item Volume propre mais peut s'ecouler
 \item Quasi incompréssible et peu dilatable
\end{enumerate}
\paragraph{Phase Gazeuse :}
\begin{enumerate}
\item Etat dispersé de la matière molécules quasi pas liées entre elles
\item Les molécules sont soumises à l'agitation thermique
\item Se dilate et se comprime
\end{enumerate}

\subsection{Etat condensé :}
\paragraph{Def :}
Etat condensé : phase solide ou liquide. Caractérisé par :

Masses volumiques :
\begin{math}
 \rho = \frac{m}{V}
\end{math}

\begin{math}
 \rho_{eau}= 10^3 kg.m^3
\end{math}

densite :
\begin{math}
 d = \frac{\rho}{\rho_{eau}}
\end{math}

\subsection{Etat fluide :}
\paragraph{Def : }Etat fluide : gaz (compressible) ou liquide (incompréssible)

or 
\begin{math}
 \rho_{air}=1,3 kg.m^{-3}
\end{math}
 et pour un gaz on a 
 \begin{math}
  d=\frac{\rho}{\rho_{air}}
 \end{math}

 \paragraph{Quantité molaire :}
 \begin{math}
  n=\frac{N}{Na}
 \end{math}

 \paragraph{Relation des gazs parfaits :}
 \begin{math}
  PV=nRT
 \end{math}
\section{Echelle d'étude}
\begin{enumerate}
 \item Echelle microscopique => atome
 \item Echelle macroscopique => au niveau du système
 \item Echelle mesoscopique => echelle intermediaire
\end{enumerate}
\section{Grandeur Thermodynamique}
\subsection{Vocabulaire}
Un système thermodynamique est défini par une surface réelle ou fictive.
\begin{enumerate}
 \item Système fermé : Pas d'échange de matière avec l'extérieur.
 \item Système ouverte : echange de matière possible.
 \item Système adiabatique : Pas d'échange thermique
 \item Système isolé : Pas d'échange d'énergie ou de matière avec l'extérieur.
\end{enumerate}
\subsection{Equilibre thermodynamique}
\paragraph{Def :}
Dans un système à l'equilibre thermodynamique les grandeur définis à l'echelle mesoscopique sont constantes.
\paragraph{Propriété:}
Tout système isolé tends vers un état d'équilibre.

\subsection{Equation d'état}
\subsubsection{Paramètre d'état}
\paragraph{Def:}
Des paramètres d'états sont des grandeurs macroscopique définissant le système et susceptible d'etre modifié lors d'une transformation.

\subsubsection{Equation d'état}
\paragraph{Def :}
La relation reliant les différents paramètres d'état est l'équation d'état.

\subsubsection{Notion de phase}
\paragraph{Def :}
Une phase est une parti d'un système et les grandeurs sont des fonctions continues de l'espace.

\section{Le gaz parfait}
\subsection{Définition :}
C'est un modéle de comportement des gazs à basse pression.

On a : 
\begin{math}
 PV=nRT
\end{math}

On peut établir cette relation au niveau théorique en supposant :
\begin{enumerate}
 \item les particulles ponctuelles
 \item les particulles son sans intéractions entre elles
 \item les éléments sont élastique
\end{enumerate}
\subsection{Amélioration du modèle}
Modèle de Van der Waals :

\begin{equation}
 (P+\frac{n^2 a}{v^2})(V-nB)=nRT
\end{equation}
b est le volume occupé par les particulles et rends compte des interractions entre les particulles.

\section{Dilatation et compréssibilité d'un système}
\end{document} 
